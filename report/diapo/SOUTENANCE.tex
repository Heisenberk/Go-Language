\documentclass{beamer}
  \usepackage[utf8]{inputenc}
  \usetheme{Warsaw}
  \graphicspath{images/}
  \usepackage{amsmath} 
  \usepackage{amssymb}

  \title{TER : Découverte du langage Go}
  \author{Clément CAUMES \& Mehdi MTALSI-MERIMI}
  \institute{UFR des Sciences Versailles - M1 Informatique}
  \date{
  	\begin{itemize}
  		\setbeamertemplate{itemize item}[default]
  		\item Réalisation d'une carte de référence
  		\item Proposition d'un ensemble d'exercices d'apprentissage
  		\item Réalisation d'une application exemple 
  	\end{itemize}
  }

  \begin{document}

\section{Carte de référence}

\subsection{Bases du langage}

\begin{frame}
\begin{block}{Bases du langage} 
	\begin{itemize}
		\setbeamertemplate{itemize item}[circle]
		\item Briques du langage (Variables, Constantes, Pointeurs...)
		\item Boucles
		\item Instructions de branchement conditionnel
		\item Instructions de branchement non conditionnel
		\item Collections et Opérations sur les collections
		\item Structure
		\item Appels de suite d'opérations (Fonctions, Méthodes et Interfaces)
		\item Ligne de commande et Arguments
	\end{itemize}
	\hspace{2.5cm}
	%\includegraphics[scale=0.3]{pics/image1.png}
\end{block}
\end{frame}

\subsection{Bibliothèques du langage Go}

\begin{frame}
\begin{block}{Principales bibliothèques standards} 
	\begin{itemize}
		\setbeamertemplate{itemize item}[circle]
		\item Bibliothèque fmt (entrées-sorties de l'utilisateur)
		\item Bibliothèque errors (gestion des exceptions)
		\item Bibliothèque os (gestion des processus, manipulation de permissions de fichiers)
		\item Bibliothèque io (interaction avec les fichiers)
		\item Bibliothèque strings (manipulation de chaînes de caractères)
		\item Bibliothèque time (mesure et gestion du temps)
		\item ...
	\end{itemize}
\end{block}

\begin{block}{Principales bibliothèques tierces} 
	\begin{itemize}
		\setbeamertemplate{itemize item}[circle]
		\item Bibliothèque debug
		\item Bibliothèque mobile
		\item ...
	\end{itemize}
	%\hspace{2.5cm}
	%\includegraphics[scale=0.3]{pics/image1.png}
\end{block}
\end{frame}

\subsection{Outils de développement}

\begin{frame}
\begin{block}{Bases du langage} 
	\begin{itemize}
		\setbeamertemplate{itemize item}[circle]
		\item Commandes (compilation, exécution, installation, tests unitaires, documentation)
		\item Compilation (création d'une application avec plusieurs packages)
		\item Tests unitaires (utilisation de tests pour vérifier des portions de code)
		\item Documentation (godoc)
		
	\end{itemize}
\end{block}
\end{frame}

\subsection{References}

\begin{frame}
\begin{block}{Références pour produire notre carte de référence} 
	\begin{itemize}
		\setbeamertemplate{itemize item}[circle]
		\item \textcolor{red}{Introduction :} \footnotesize \url{https://fr.wikipedia.org/wiki/Go_(langage)} \normalsize
		\item \textcolor{red}{Site officiel Golang :} \footnotesize \url{https://golang.org/} \normalsize
		\item \textcolor{red}{Bases du langage Go :} \footnotesize \url{https://www.tutorialspoint.com/go/index.htm} \normalsize
		\item \textcolor{red}{Illustrations de code Go :} \footnotesize \url{https://gobyexample.com/} \normalsize
		\item \textcolor{red}{Goroutines :} \footnotesize \url{https://blog.fedora-fr.org/metal3d/post/Go-et-les-goroutines-introduction-au-langage} \normalsize
		\item \textcolor{red}{Go vs C++ :} \footnotesize \url{https://www.scriptol.fr/programmation/go.php} \normalsize
		\item \textcolor{red}{Application Android :} \footnotesize \url{https://play.google.com/store/apps/details?id=in.intelitech.golang&hl=en_US} \normalsize
	\end{itemize}
\end{block}
\end{frame}

      
\end{document}
