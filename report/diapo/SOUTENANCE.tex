\documentclass{beamer}
  \usepackage[utf8]{inputenc}
  \usetheme{Warsaw}
  \graphicspath{images/}
  \usepackage{amsmath} 
  \usepackage{amssymb}

  \title{TER : Découverte du langage Go}
  \author{Clément CAUMES \& Mehdi MTALSI-MERIMI}
  \institute{UFR des Sciences Versailles - M1 Informatique}
  \date{
  	\begin{itemize}
  		\setbeamertemplate{itemize item}[default]
  		\item Réalisation d'une carte de référence
  		\item Proposition d'un ensemble d'exercices d'apprentissage
  		\item Réalisation d'une application exemple 
  	\end{itemize}
  }

  \begin{document}
  	
  \begin{frame}
  	\titlepage
  \end{frame}

\section{Carte de référence}

\subsection{Bases du langage}

\begin{frame}
\begin{block}{Bases du langage} 
	\begin{itemize}
		\setbeamertemplate{itemize item}[circle]
		\item Briques du langage (Variables, Constantes, Pointeurs...)
		\item Boucles
		\item Instructions de branchement conditionnel
		\item Instructions de branchement non conditionnel
		\item Collections et Opérations sur les collections
		\item Structure
		\item Appels de suite d'opérations (Fonctions, Méthodes et Interfaces)
		\item Ligne de commande et Arguments
	\end{itemize}
	\hspace{2.5cm}
	%\includegraphics[scale=0.3]{pics/image1.png}
\end{block}
\end{frame}

\subsection{Bibliothèques du langage Go}

\begin{frame}
\begin{block}{Principales bibliothèques standards} 
	\begin{itemize}
		\setbeamertemplate{itemize item}[circle]
		\item Bibliothèque 'fmt' (entrées-sorties de l'utilisateur)
		\item Bibliothèque 'errors' (gestion des exceptions)
		\item Bibliothèque 'os' (gestion des processus, manipulation de permissions de fichiers)
		\item Bibliothèque 'io' (interaction avec les fichiers)
		\item Bibliothèque 'strings' (manipulation de chaînes de caractères)
		\item Bibliothèque 'time' (mesure et gestion du temps)
		\item ...
	\end{itemize}
\end{block}

\begin{block}{Principales bibliothèques tierces} 
	\begin{itemize}
		\setbeamertemplate{itemize item}[circle]
		\item Bibliothèque 'debug'
		\item Bibliothèque 'mobile'
		\item ...
	\end{itemize}
	%\hspace{2.5cm}
	%\includegraphics[scale=0.3]{pics/image1.png}
\end{block}
\end{frame}

\subsection{Outils de développement}

\begin{frame}
\begin{block}{Outils}
	\begin{itemize}
		\setbeamertemplate{itemize item}[circle]
		\item Commandes (compilation, exécution, installation, tests unitaires, documentation)
		\item Compilation (création d'une application avec plusieurs packages)
		\item Tests unitaires (utilisation de tests pour vérifier des portions de code)
		\item Documentation (godoc)
		
	\end{itemize}
\end{block}
\end{frame}

\subsection{References}

\begin{frame}
\begin{block}{Références pour produire notre carte de référence} 
	\begin{itemize}
		\setbeamertemplate{itemize item}[circle]
		\item \textcolor{red}{Introduction :} \footnotesize \url{https://fr.wikipedia.org/wiki/Go_(langage)} \normalsize
		\item \textcolor{red}{Site officiel Golang :} \footnotesize \url{https://golang.org/} \normalsize
		\item \textcolor{red}{Bases du langage Go :} \footnotesize \url{https://www.tutorialspoint.com/go/index.htm} \normalsize
		\item \textcolor{red}{Illustrations de code Go :} \footnotesize \url{https://gobyexample.com/} \normalsize
		\item \textcolor{red}{Goroutines :} \footnotesize \url{https://blog.fedora-fr.org/metal3d/post/Go-et-les-goroutines-introduction-au-langage} \normalsize
		\item \textcolor{red}{Go vs C++ :} \footnotesize \url{https://www.scriptol.fr/programmation/go.php} \normalsize
		\item \textcolor{red}{Application Android :} \footnotesize \url{https://play.google.com/store/apps/details?id=in.intelitech.golang&hl=en_US} \normalsize
	\end{itemize}
\end{block}
\end{frame}

\section{Exercices d'apprentissage}

\subsection{TD1 : Installation \& Manipulation des bases du langage}

\begin{frame}
\begin{block}{TD1 : Installation \& Manipulation des bases du langage} 
	\begin{itemize}
		\setbeamertemplate{itemize item}[circle]
		\item \textcolor{blue}{Exercice 1} : Installation de l'environnement Golang et réalisation du premier programme Go 
		\item \textcolor{blue}{Exercice 2} : Calcul de conversions d'un temps-secondes en temps exprimé en heures, minutes et secondes
		\item \textcolor{blue}{Exercice 3} : Réalisation de fonctions récursives (Fibonacci et factorielle)
		\item \textcolor{blue}{Exercice 4} : Test naïf de primalité
		\item \textcolor{blue}{Exercice 5} : Test de nombres amicaux
		\item \textcolor{blue}{Exercice 6} : Création d'une structure de données pour représenter une fraction et réalisation des fonctions de calculs de fractions
	\end{itemize}
\end{block}
\end{frame}

\subsection{TD2 : Manipulation de structures complexes}

\begin{frame}
\begin{block}{TD2 : Manipulation de structures complexes} 
	\begin{itemize}
		\setbeamertemplate{itemize item}[circle]
		\item \textcolor{blue}{Exercice 1} : Implémentation de tris (à bulles et fusion) de tableaux
		\item \textcolor{blue}{Exercice 2} : Implémentation d'un annuaire électronique et première utilisation de méthodes
		\item \textcolor{blue}{Exercice 3} : Implémentation de listes et de méthodes manipulant ces dernières 
	\end{itemize}
\end{block}
\end{frame}

\subsection{TD3 : Manipulation avancée de bibliothèques Go}

\begin{frame}
\begin{block}{TD3 : Manipulation avancée de bibliothèques Go} 
	\begin{itemize}
		\setbeamertemplate{itemize item}[circle]
		\item \textcolor{blue}{Exercice 1} : Manipulation de la bibliothèque 'image' pour dessiner un damier, puis une fractale de Mandelbrot
		\item \textcolor{blue}{Exercice 2} : Manipulation de la bibliothèque 'io' pour implémenter une fonction qui copie le contenu d'un fichier dans un autre fichier et une fonction de lecture des métadonnées d'un fichier
	\end{itemize}
\end{block}
\end{frame}

\subsection{TD4 : Manipulation des outils de développement}

\begin{frame}
\begin{block}{TD4 : Manipulation des outils de développement} 
	\begin{itemize}
		\setbeamertemplate{itemize item}[circle]
		\item \textcolor{blue}{Exercice 1} : Implémentation et manipulation de polynômes en utilisant les outils de développement (compilation, plusieurs packages, documentation godoc, tests unitaires, gestion des exceptions)
	\end{itemize}
\end{block}
\end{frame}

\section{Application exemple}

\subsection{Présentation}

\begin{frame}
\begin{block}{Application exemple : goshield} 
	Il est primordial de prendre conscience de l'importance de chiffrer ses propres communications privées. En effet, avec l'émergence des réseaux informatiques, il y a de plus en plus de risques d'avoir ses communications personnelles surveillées. D'où l'importance d'utiliser des applications de chiffrement.
	Le but est de créer une application en ligne de commande pour le chiffrement et le déchiffrement de fichiers/dossiers. Son utilisation pourrait être, par exemple, de chiffrer le contenu de dépôt git afin de le rendre illisible pour le public. 
\end{block}

\begin{alertblock}{Contraintes} 
	\begin{itemize}
		\setbeamertemplate{itemize item}[circle]
		\item utilisation des outils de développement acquis lors de la lecture de la carte de référence (tests unitaires, compilation de packages, documentation godoc ...)
	\end{itemize}
\end{alertblock}
\end{frame}

\begin{frame}
\begin{exampleblock}{Exemple d'utilisation}
	L'application propose deux outils différents :  
	\begin{itemize}
		\setbeamertemplate{itemize item}[circle]
		\item chiffrer une liste de fichiers/dossiers avec la commande : \textbf{goshield --encrypt -p projet dossier1/sous-dossier2/ fichier1.txt image.png}
		\item déchiffrer une liste de fichiers/dossiers avec la commande : \textbf{./goshield --decrypt -p projet dossier1/sous-dossier2/ fichier1.txt.gsh image.png.gsh}
	\end{itemize}
\end{exampleblock}
\end{frame}


\begin{frame}
\begin{block}{Définitions} 
	L’algorithme de chiffrement choisi pour goshield est l’AES (Advanced Encryption Standard). C’est l’un des algorithmes symétriques les plus sécurisés puisqu’aucune attaque n’a été démontrée (mise à part l’attaque par force brute qui n’est pas réalisable avec la puissance de calcul actuelle).
	
	GoShield proposera AES-256 (avec une clé sur 256 bits pour être le plus sécurisé possible). 
	
	De plus, l'utilisation du mode opératoire CBC est le plus efficace afin de chiffrer un fichier de taille variable. 
\end{block}
\end{frame}

\subsection{Fonctionnement}

\subsubsection{Chiffrement}

\begin{frame}
\begin{block}{Etapes du chiffrement goshield} 
	Le chiffrement d’un fichier consiste en plusieurs étapes :
	\begin{itemize}
	\setbeamertemplate{itemize item}[circle]
	\item écriture de la signature GOSHIELD.
	\item génération et écriture du sel cryptographique pseudo-aléatoire.
	\item génération et écriture du vecteur d’initialisation (IV).
	\item écriture de la taille du dernier bloc en octets.
	\item chiffrement et écriture de chaque bloc chiffré en utilisant AES-256 avec CBC. Le contenu de ce chiffrement sera écrit dans un fichier avec le nom du clair initial concaténé à l’extension goshield (.gsh).
	\end{itemize}
	
\end{block}
\end{frame}

\subsubsection{Déchiffrement}

\begin{frame}
\begin{block}{Etapes du déchiffrement goshield} 
	Le déchiffrement d’un fichier consiste en plusieurs étapes :
	\begin{itemize}
		\setbeamertemplate{itemize item}[circle]
		
		\item vérification de la bonne extension .gsh.
		\item vérification de la signature GOSHIELD censée apparaître sur les 8 premiers octets.
		\item lecture du sel cryptographique et calcul de la clé en concaténant le sel avec le mot de passe choisi par l’utilisateur lors du déchiffrement. 
		\item lecture de la valeur du vecteur d’initialisation (IV).
		\item lecture de la taille du dernier bloc en octets. Cela permettra d’enlever le padding sur le dernier bloc.
		\item déchiffrement et écriture de chaque bloc déchiffré en utilisant AES-256 avec CBC. 
	\end{itemize}
	
\end{block}
\end{frame}

\subsection{Conclusion}

\begin{frame}
\begin{block}{Pourquoi avoir choisi le langage Go pour cette application ?} 
	Le langage Go présente un intêret pour ce type d'application :
	\begin{itemize}
		\setbeamertemplate{itemize item}[circle]
		
		\item il s'agit d'un langage facile à comprendre et qui est aussi puissant que certains langages bas niveaux tel que le langage C.
		\item Go est également très intéressant pour la programmation "multithread" qui a été utilisée pour cette application. En effet, la gestion de concurrence peut être maintenue aisément grâce à sa facilité d'utilisation. 
		\item Golang propose de nombreuses bibliothèques. Pour l'application goshield, la bibliothèque crypto nous a permis de travailler sur le chiffrement AES. 
	\end{itemize}
	
\end{block}
\end{frame}
      
\end{document}
