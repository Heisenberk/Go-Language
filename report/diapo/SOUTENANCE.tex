\documentclass{beamer}
  \usepackage[utf8]{inputenc}
  \usetheme{Warsaw}
  \graphicspath{images/}
  \usepackage{amsmath} 
  \usepackage{amssymb}

  \title{TER : Découverte du langage Go}
  \author{Clément CAUMES \& Mehdi MTALSI-MERIMI}
  \institute{UFR des Sciences Versailles - M1 Informatique}
  \date{
  	\begin{itemize}
  		\setbeamertemplate{itemize item}[default]
  		\item Réalisation d'une carte de référence
  		\item Proposition d'un ensemble d'exercices d'apprentissage
  		\item Réalisation d'une application exemple 
  	\end{itemize}
  }

  \begin{document}
  	
  \begin{frame}
  	\titlepage
  \end{frame}

\section{Carte de référence}

\subsection{Bases du langage}

\begin{frame}
\begin{block}{Bases du langage} 
	\begin{itemize}
		\setbeamertemplate{itemize item}[circle]
		\item Briques du langage (Variables, Constantes, Pointeurs...)
		\item Boucles
		\item Instructions de branchement conditionnel
		\item Instructions de branchement non conditionnel
		\item Collections et Opérations sur les collections
		\item Structure
		\item Appels de suite d'opérations (Fonctions, Méthodes et Interfaces)
		\item Ligne de commande et Arguments
	\end{itemize}
	\hspace{2.5cm}
	%\includegraphics[scale=0.3]{pics/image1.png}
\end{block}
\end{frame}

\subsection{Bibliothèques du langage Go}

\begin{frame}
\begin{block}{Principales bibliothèques standards} 
	\begin{itemize}
		\setbeamertemplate{itemize item}[circle]
		\item Bibliothèque 'fmt' (entrées-sorties de l'utilisateur)
		\item Bibliothèque 'errors' (gestion des exceptions)
		\item Bibliothèque 'os' (gestion des processus, manipulation de permissions de fichiers)
		\item Bibliothèque 'io' (interaction avec les fichiers)
		\item Bibliothèque 'strings' (manipulation de chaînes de caractères)
		\item Bibliothèque 'time' (mesure et gestion du temps)
		\item ...
	\end{itemize}
\end{block}

\begin{block}{Principales bibliothèques tierces} 
	\begin{itemize}
		\setbeamertemplate{itemize item}[circle]
		\item Bibliothèque 'debug'
		\item Bibliothèque 'mobile'
		\item ...
	\end{itemize}
	%\hspace{2.5cm}
	%\includegraphics[scale=0.3]{pics/image1.png}
\end{block}
\end{frame}

\subsection{Outils de développement}

\begin{frame}
\begin{block}{Bases du langage} 
	\begin{itemize}
		\setbeamertemplate{itemize item}[circle]
		\item Commandes (compilation, exécution, installation, tests unitaires, documentation)
		\item Compilation (création d'une application avec plusieurs packages)
		\item Tests unitaires (utilisation de tests pour vérifier des portions de code)
		\item Documentation (godoc)
		
	\end{itemize}
\end{block}
\end{frame}

\subsection{References}

\begin{frame}
\begin{block}{Références pour produire notre carte de référence} 
	\begin{itemize}
		\setbeamertemplate{itemize item}[circle]
		\item \textcolor{red}{Introduction :} \footnotesize \url{https://fr.wikipedia.org/wiki/Go_(langage)} \normalsize
		\item \textcolor{red}{Site officiel Golang :} \footnotesize \url{https://golang.org/} \normalsize
		\item \textcolor{red}{Bases du langage Go :} \footnotesize \url{https://www.tutorialspoint.com/go/index.htm} \normalsize
		\item \textcolor{red}{Illustrations de code Go :} \footnotesize \url{https://gobyexample.com/} \normalsize
		\item \textcolor{red}{Goroutines :} \footnotesize \url{https://blog.fedora-fr.org/metal3d/post/Go-et-les-goroutines-introduction-au-langage} \normalsize
		\item \textcolor{red}{Go vs C++ :} \footnotesize \url{https://www.scriptol.fr/programmation/go.php} \normalsize
		\item \textcolor{red}{Application Android :} \footnotesize \url{https://play.google.com/store/apps/details?id=in.intelitech.golang&hl=en_US} \normalsize
	\end{itemize}
\end{block}
\end{frame}

\section{Exercices d'apprentissage}

\subsection{TD1 : Installation \& Manipulation des bases du langage}

\begin{frame}
\begin{block}{TD1 : Installation \& Manipulation des bases du langage} 
	\begin{itemize}
		\setbeamertemplate{itemize item}[circle]
		\item \textcolor{blue}{Exercice 1} : Installation de l'environnement Golang et réalisation du premier programme Go 
		\item \textcolor{blue}{Exercice 2} : Calcul de conversions d'un temps-secondes en temps exprimé en heures, minutes et secondes
		\item \textcolor{blue}{Exercice 3} : Réalisation de fonctions récursives (Fibonacci et factorielle)
		\item \textcolor{blue}{Exercice 4} : Test naïf de primalité
		\item \textcolor{blue}{Exercice 5} : Test de nombres amicaux
		\item \textcolor{blue}{Exercice 6} : Création d'une structure de données pour représenter une fraction et réalisation des fonctions de calculs de fractions
	\end{itemize}
\end{block}
\end{frame}

\subsection{TD2 : Manipulation de structures complexes}

\begin{frame}
\begin{block}{TD2 : Manipulation de structures complexes} 
	\begin{itemize}
		\setbeamertemplate{itemize item}[circle]
		\item \textcolor{blue}{Exercice 1} : Implémentation de tris (à bulles et fusion) de tableaux
		\item \textcolor{blue}{Exercice 2} : Implémentation d'un annuaire électronique et première utilisation de méthodes
		\item \textcolor{blue}{Exercice 3} : Implémentation de listes et de méthodes manipulant ces dernières 
	\end{itemize}
\end{block}
\end{frame}

\subsection{TD3 : Manipulation avancée de bibliothèques Go}

\begin{frame}
\begin{block}{TD3 : Manipulation avancée de bibliothèques Go} 
	\begin{itemize}
		\setbeamertemplate{itemize item}[circle]
		\item \textcolor{blue}{Exercice 1} : Manipulation de la bibliothèque 'image' pour dessiner un damier, puis une fractale de Mandelbrot
		\item \textcolor{blue}{Exercice 2} : Manipulation de la bibliothèque 'io' pour implémenter une fonction qui copie le contenu d'un fichier dans un autre fichier et une fonction de lecture des métadonnées d'un fichier
	\end{itemize}
\end{block}
\end{frame}

\subsection{TD4 : Manipulation des outils de développement}

\begin{frame}
\begin{block}{TD4 : Manipulation des outils de développement} 
	\begin{itemize}
		\setbeamertemplate{itemize item}[circle]
		\item \textcolor{blue}{Exercice 1} : Implémentation et manipulation de polynômes en utilisant les outils de développement (compilation, plusieurs packages, documentation godoc, tests unitaires, gestion des exceptions)
	\end{itemize}
\end{block}
\end{frame}

\section{Application exemple}

\subsection{Présentation}

\begin{frame}
\begin{block}{Introduction} 
	Il est primordial de prendre conscience de l'importance de chiffrer ses propres communications privées. En effet, avec l'émergence des réseaux informatiques, il y a de plus en plus de risques d'avoir ses communications personnelles surveillées. D'où l'importance d'utiliser des applications de chiffrement.
	Le but est de créer une application en ligne de commande pour le chiffrement et le déchiffrement de fichiers/dossiers. Son utilisation pourrait être, par exemple, de chiffrer le contenu de dépôt git afin de le rendre illisible pour le public. 
\end{block}

\begin{alertblock}{Contraintes} 
	\begin{itemize}
		\setbeamertemplate{itemize item}[circle]
		\item utilisation des outils de développement acquis lors de la lecture de la carte de référence (tests unitaires, compilation de packages, documentation godoc ...)
	\end{itemize}
\end{alertblock}
\end{frame}

\subsection{Fonctionnement}

\subsection{Conclusion}

% pk utiliser l'appli, pk utiliser Go ...

      
\end{document}
