\documentclass[11pt]{article}
%\documentclass{book}
\usepackage[utf8]{inputenc}
\usepackage[T1]{fontenc}
\usepackage[french]{babel}
\usepackage[top=1.8cm, bottom=1.8cm, left=1.8cm, right=1.8cm]{geometry}
\usepackage[linktocpage,colorlinks=false]{hyperref}
\usepackage{graphicx}
\usepackage{epsfig}
\usepackage{amssymb}
\usepackage{amsmath}
\usepackage{array}
\usepackage{subfig}
\usepackage{multicol}
\usepackage{caption}
\usepackage{algorithm}
\usepackage{algorithmic}
\hypersetup{
    colorlinks=true,
    breaklinks=true,
    urlcolor=blue,
}
\parskip=5pt

\title{\huge{\textbf {Carte de référence du langage Go}}}
\author{CAUMES Clément \\ MTALSI-MERIMI Mehdi}
\date{}

\begin{document}

\maketitle
\vspace{20em}
\begin{figure}
\centering
\includegraphics[scale=0.2]{pic/logo.jpg}
\end{figure}
\newpage

\tableofcontents
\newpage

\section{Préambule}

\subsection{Présentation générale}

Le langage Go est un langage de programmation compilé, c'est-à-dire que le code source du programme est transformé en code machine par le compilateur. Il suit également le paradigme de programmation concurrent, ce qui signifie que l'exécution du programme se fait sur une pile de threads et de processus. C'est un langage impératif : cela signifie qu'une suite d'instructions modifie l'état du programme. 

Le langage Go a été écrit en langage C, en Yacc et en Bison (pour le parser permettant ainsi de définir des règles de syntaxe dans le langage Go). Il a été inventé par Robert Griesemer, Rob Pike et Ken Thompson. Sa première version date de novembre 2009. 

Le but du langage Go est de proposer un langage simple à comprendre et facile à utiliser (d'où son influence de Python) mais permettant de faire des programmes complexes (influence du C). 


\section{Bases du langage Go}

\subsection{Composants élémentaires de Go}

\subsection{Structure d'un programme}

\subsection{Types prédéfinis}

\subsection{Constantes}

\subsection{Opérateurs}

\subsection{Instructions de branchement conditionnel}

\subsection{Boucles}

\subsection{Instructions de branchement non conditionnel}

\subsection{Fonctions d'entrées-sorties}

\section{Types composés}

\section{Fonctions}

\section{Bibliothèques du langage Go}

\section{Directives au préprocesseur}

\section{Programmation modulaire}



\end{document}
